\documentclass[11pt]{beamer}
\usetheme{Madrid}
\usefonttheme{serif}

\usepackage[utf8]{inputenc}
% \usepackage[brazil]{babel}
\usepackage[T1]{fontenc}

\usepackage{amsmath}
\usepackage{amsfonts}
\usepackage{amssymb}
\usepackage{graphicx}

\usepackage{caption}
\usepackage{siunitx}

\DeclareMathOperator{\sen}{sen}
\DeclareMathOperator{\tg}{tg}

\setbeamertemplate{caption}[numbered]

\author[]{Diogo Fonseca \\ Professor: João Dias}
\title{From Waveforms to Bits}
% Informe o seu email de contato no comando a seguir
% Por exemplo, alcebiades.col@ufes.br
\newcommand{\email}{email}
%\setbeamercovered{transparent} 
\setbeamertemplate{navigation symbols}{} 
\logo{\includegraphics[scale=0.15]{imagens/ualg_tiny.png}}
\institute[]{UALG \par MESTRADO EM ENGENHARIA INFORMÁTICA} 
\date{8 de Novembro de 2025}
%\subject{}

% ---------------------------------------------------------
% Selecione um estilo de referência
\bibliographystyle{apalike}

%\bibliographystyle{abbrv}
%\setbeamertemplate{bibliography item}{\insertbiblabel}
% ---------------------------------------------------------

% ---------------------------------------------------------
% Incluir os slides nos quais as referências foram citadas
%\usepackage[brazilian,hyperpageref]{backref}

%\renewcommand{\backrefpagesname}{Citado na(s) página(s):~}
%\renewcommand{\backref}{}
%\renewcommand*{\backrefalt}[4]{
%	\ifcase #1 %
%		Nenhuma citação no texto.%
%	\or
%		Citado na página #2.%
%	\else
%		Citado #1 vezes nas páginas #2.%
%	\fi}%
% ---------------------------------------------------------

\begin{document}

\begin{frame}
	\titlepage
\end{frame}

\begin{frame}{Table of contents}
    \only<1>{\tableofcontents[sections={1-3}]}
    \only<2>{\tableofcontents[sections={4-6}]}
\end{frame}


%\begin{frame}[allowframebreaks]{Sumário}
%\tableofcontents 
%\end{frame}



\section{Theoretical Basis}

%%%%%%%%%%%%%%%%%%%%%%%%%%%%%%%%%%%%%%%%%%%%%%%%%%%%%%%%%%%%%%
\subsection{Fourier Analysis}

\begin{frame}
    \begin{beamercolorbox}[sep=8pt,center,rounded=true,shadow=true]{title}
        \usebeamerfont{title}\Huge\textbf{Fourier Analysis}
    \end{beamercolorbox}
\end{frame}

\begin{frame}{Fourier Analysis}
    \begin{block}{Fourier Series}
        $$ g(t) = \frac{1}{2}\boldsymbol{c} + \sum^{\infty}_{n=1} \boldsymbol{a_n} sin(2 \pi n f t) + \sum^{\infty}_{n=1} \boldsymbol{b_n} cos(2 \pi n f t)$$
    \end{block}

    \bigskip
    
	\only<2->{$ \boldsymbol{a_n} = \frac{2}{T} \int^{T}_{0} g(t) sin(2 \pi n f t) dt $}

    \bigskip
    
	\only<3->{$ \boldsymbol{b_n} = \frac{2}{T} \int^{T}_{0} g(t) cos(2 \pi n f t) dt $}

    \bigskip
    
	\only<4->{$ \boldsymbol{c} = \frac{2}{T} \int^{T}_{0} g(t)dt $}
\end{frame}







%%%%%%%%%%%%%%%%%%%%%%%%%%%%%%%%%%%%%%%%%%%%%%%%%%%%%%%%%%%%%%
\subsection{Bandwidth-Limited Signals}

\begin{frame}
    \begin{beamercolorbox}[sep=8pt,center,rounded=true,shadow=true]{title}
        \usebeamerfont{title}\Huge\textbf{Bandwidth-Limited Signals}
    \end{beamercolorbox}
\end{frame}

\begin{frame}{Bandwidth-Limited Signals}
	\begin{center}
		\color{structure}
		\bfseries\LARGE
		Some definitions
	\end{center}

	\vspace{1.0cm}

	\begin{itemize}
		\item Bandwidth (analog)
		\item Bandwidth (digital)
		\item Cutoff
		\item Baseband
		\item Passband
	\end{itemize}
\end{frame}

\begin{frame}{Bandwidth-Limited Signals}
    \begin{figure}[htb]
        \centering

        \includegraphics[width=0.40\textwidth]{imagens/harmonics.png}

		\captionsetup{font=scriptsize}
		\caption{(a) A binary signal and its root-mean-square Fourier amplitudes. (b)-(e) Successive approximations to the original signal.}
    \end{figure}
\end{frame}

\begin{frame}{Bandwidth-Limited Signals}
    \begin{table}[htb]
        \centering

		\begin{tabular}{|r|S|S|S|}
	        \hline
			\textbf{Bps} & \textbf{T (msec)} & \textbf{First harmonic (Hz)} & \textbf{\# Harmonics sent} \\ \hline
	        300		& 26.67	& 37.5	& 80	\\ \hline
	        600		& 13.33	& 75	& 40	\\ \hline
	        1200	& 6.67	& 150	& 20	\\ \hline
	        2400	& 3.33	& 300	& 10	\\ \hline
	        4800	& 1.67	& 600	& 5		\\ \hline
	        9600	& 0.83	& 1200	& 2		\\ \hline
	        19200	& 0.42	& 2400	& 1		\\ \hline
	        38400	& 0.21	& 4800	& 0		\\ \hline
        \end{tabular}

		\captionsetup{font=scriptsize}
		\caption{Relation between data rate and harmonics.}
    \end{table}
\end{frame}




%%%%%%%%%%%%%%%%%%%%%%%%%%%%%%%%%%%%%%%%%%%%%%%%%%%%%%%%%%%%%%
\section{The Maximum Data Rate of a Channel}

\begin{frame}
    \begin{beamercolorbox}[sep=8pt,center,rounded=true,shadow=true]{title}
        \usebeamerfont{title}\Huge\textbf{Maximum Data Rate of a Channel}
    \end{beamercolorbox}
\end{frame}

\begin{frame}{Maximum Data Rate of a Channel}
	Assume a perfect channel for communication with \textbf{no noise or interferance}.
	Furthermore, assume an arbitrary signal that has been run through a low-pass filter of bandwidth $\boldsymbol{B}$.
	\begin{block}<2->{Nyquist's sample rate proof}
		The filtered signal can be completely reconstructed by making exactly $\boldsymbol{2B}$ samples per second.
    \end{block}
	\only<3->{And given $\boldsymbol{V}$, the number of discrete states a signal can transmit at each sampling instant:}
	\begin{block}<4->{Nyquist's theorem}
		Maximum data rate of channel = $ \boldsymbol{2B \log_2(V)}$ bits/sec
    \end{block}
\end{frame}

\begin{frame}{Maximum Data Rate of a Channel}
	\centering
	\usebeamerfont{title}\Huge\textbf{What about interferance?}
\end{frame}

\begin{frame}{Maximum Data Rate of a Channel}
	Assume a channel analogous to the last, with a signal power $ \boldsymbol{S} $ and a noise power $ \boldsymbol{N} $.
	\begin{block}<2->{SNR (Signal-to-Noise Ratio)}
		$$ \frac{\boldsymbol{S}}{\boldsymbol{N}} $$
    \end{block}
	\begin{block}<3->{Capacity of a real channel}
		Maximum data rate of channel = $ \boldsymbol{B \log_2(1 + \frac{S}{N})}$ bits/sec
    \end{block}
\end{frame}






%%%%%%%%%%%%%%%%%%%%%%%%%%%%%%%%%%%%%%%%%%%%%%%%%%%%%%%%%%%%%%
\section{Digital Modulation}

\begin{frame}
    \begin{beamercolorbox}[sep=8pt,center,rounded=true,shadow=true]{title}
        \usebeamerfont{title}\Huge\textbf{Digital Modulation}
    \end{beamercolorbox}
\end{frame}



%%%%%%%%%%%%%%%%%%%%%%%%%%%%%%%%%%%%%%%%%%%%%%%%%%%%%%%%%%%%%%
\subsection{Baseband Transmission}

\begin{frame}
    \begin{beamercolorbox}[sep=8pt,center,rounded=true,shadow=true]{title}
        \usebeamerfont{title}\Huge\textbf{Baseband transmission}
    \end{beamercolorbox}
\end{frame}

\begin{frame}{Baseband Transmission}
	\textbf{NRZ scheme} (Non-Return-to-Zero)
	\begin{itemize}
		\item<2-> Signal follows the data. (i.e. Positive voltage for one, negative voltage for zero)
		\item<2-> Sender sends the signal.
		\item<2-> Receiver samples the signal at regular intervals of time.
		\item<2-> Receiver maps signal samples to the closes symbols (i.e. $\{0, 1\}$)
	\end{itemize}
	\includegraphics[width=1.00\textwidth]{imagens/NRZ.png}<3->
\end{frame}

%%%%%%%%%%%%%%%%%%%%%%%%%%%%%%%%%%%%%%%%%%%%%%%%%%%%%%%%%%%%%%
\subsection{Bandwidth Efficiency}

\begin{frame}
    \begin{beamercolorbox}[sep=8pt,center,rounded=true,shadow=true]{title}
        \usebeamerfont{title}\Huge\textbf{Bandwidth Efficiency}
    \end{beamercolorbox}
\end{frame}

\begin{frame}{Bandwidth Efficiency}
	\textbf{NRZ scheme}
	\begin{itemize}
		\item<2-> For a data rate of $ B $ bits/sec we need a bandwidth of at least $ \boldsymbol{\frac{B}{2}} $ Hz.
		\item<3-> Given this is a fundamental limit, we cannot run NRZ faster without using additional bandwidth.
		\item<4-> One strategy to increase efficiency is to use more than two signaling levels.
		\item<5-> By using 4 different voltages, for instance, we can send 2 bits at once as a single \textbf{symbol}.
		\item<6-> This works as long as the signal at the receiver is sufficiently strong to distinguish all the levels.
		\item<7-> The rate at which the signal changes (\textbf{symbol rate}) is then half the bit rate, so bandwidth has been reduced.
		\item<8-> Bitrate can be interpreted as \textbf{(symbol rate $ \boldsymbol{\times} $ bits per symbol)}.
	\end{itemize}
\end{frame}


%%%%%%%%%%%%%%%%%%%%%%%%%%%%%%%%%%%%%%%%%%%%%%%%%%%%%%%%%%%%%%
\subsection{Clock Recovery}

\begin{frame}
    \begin{beamercolorbox}[sep=8pt,center,rounded=true,shadow=true]{title}
        \usebeamerfont{title}\Huge\textbf{Clock Recovery}
    \end{beamercolorbox}
\end{frame}

\begin{frame}{Clock Recovery}
	\begin{center}
		\color{structure}
		\bfseries\LARGE
		The problem
	\end{center}

	\begin{itemize}
		\item<2-> The receiver must know when one symbol ends and the next symbol appears.
		\item<3-> After a while it's hard to tell the bits apart, 15 zeroes look much like 16 zeroes unless you have a very accurate clock.
		\item<4-> Very accurate clocks are expensive.
	\end{itemize}
\end{frame}

\begin{frame}{Clock Recovery}
	\centering
	\textbf{Manchester encoding}

	\bigskip

	\includegraphics[width=1.00\textwidth]{imagens/bit_stream.png}
	\includegraphics[width=1.00\textwidth]{imagens/manchester_encoding.png}

	\begin{itemize}
		\item<2-> Requires twice the bandwidth.
		\item<2-> Used in classic Ethernet.
	\end{itemize}
\end{frame}

\begin{frame}{Clock Recovery}
	\centering
	\textbf{NRZI (Non-Return-to-Zero Inverse)}

	\bigskip

	\includegraphics[width=1.00\textwidth]{imagens/bit_stream.png}
	\includegraphics[width=1.00\textwidth]{imagens/NRZI.png}

	\begin{itemize}
		\item<2-> Long streaks of 0 still have the same problem.
		\item<2-> Used in USB.
	\end{itemize}
\end{frame}

\begin{frame}{Clock Recovery}
	\centering
	\textbf{4B/5B}

	\bigskip

    \begin{table}[htb]
        \centering

		\begin{tabular}{|l|l|l|l|}
	        \hline
			\textbf{Data (4B)} & \textbf{Codeword (5B)} & \textbf{Data (4B)} & \textbf{Codeword (5B)} \\ \hline
	        0000	& 11110	& 1000	& 10010	\\ \hline
	        0001	& 01001	& 1001	& 10011	\\ \hline
	        0010	& 10100	& 1010	& 10110	\\ \hline
	        0011	& 10101	& 1011	& 10111	\\ \hline
	        0100	& 01010	& 1100	& 11010	\\ \hline
	        0101	& 01011	& 1101	& 11011	\\ \hline
	        0110	& 01110	& 1110	& 11100	\\ \hline
	        0111	& 01111	& 1111	& 11101	\\ \hline
        \end{tabular}

		\captionsetup{font=scriptsize}
		\caption{4B/5B mapping.}
    \end{table}

	\begin{itemize}
		\item<2-> 25\% overhead.
	\end{itemize}
\end{frame}

\begin{frame}{Clock Recovery}
	\begin{center}
		\textbf{Scrambling}
	\end{center}

	\bigskip

	Scrambles the data by XORing it with a pseudorandom sequence. Receiver XORs the data with the same sequence.

	\bigskip

	\begin{itemize}
		\item<2-> Used in early versions of IP over SONET.
		\item<2-> Still possible to get long streaks of identical symbols.
		\item<2-> Possible malicious usage, "killer packets".
	\end{itemize}
\end{frame}

%%%%%%%%%%%%%%%%%%%%%%%%%%%%%%%%%%%%%%%%%%%%%%%%%%%%%%%%%%%%%%
\subsection{Balanced Signals}

\begin{frame}
    \begin{beamercolorbox}[sep=8pt,center,rounded=true,shadow=true]{title}
        \usebeamerfont{title}\Huge\textbf{Balanced Signals}
    \end{beamercolorbox}
\end{frame}

%%%%%%%%%%%%%%%%%%%%%%%%%%%%%%%%%%%%%%%%%%%%%%%%%%%%%%%%%%%%%%
\subsection{Passband Transmission}

%%%%%%%%%%%%%%%%%%%%%%%%%%%%%%%%%%%%%%%%%%%%%%%%%%%%%%%%%%%%%%
\section{Multiplexing}
%%%%%%%%%%%%%%%%%%%%%%%%%%%%%%%%%%%%%%%%%%%%%%%%%%%%%%%%%%%%%%
\subsection{FDM (Frequency Division Multiplexing)}
%%%%%%%%%%%%%%%%%%%%%%%%%%%%%%%%%%%%%%%%%%%%%%%%%%%%%%%%%%%%%%
\subsection{OFDM (Orthogonal Frequency Division Multiplexing)}
%%%%%%%%%%%%%%%%%%%%%%%%%%%%%%%%%%%%%%%%%%%%%%%%%%%%%%%%%%%%%%
\subsection{TDM (Time Division Multiplexing)}
%%%%%%%%%%%%%%%%%%%%%%%%%%%%%%%%%%%%%%%%%%%%%%%%%%%%%%%%%%%%%%
\subsection{STDM (Statistical Time Division Multiplexing)}
%%%%%%%%%%%%%%%%%%%%%%%%%%%%%%%%%%%%%%%%%%%%%%%%%%%%%%%%%%%%%%
\subsection{CDM (Code Division Multiplexing)}
%%%%%%%%%%%%%%%%%%%%%%%%%%%%%%%%%%%%%%%%%%%%%%%%%%%%%%%%%%%%%%
\subsection{CDMA (Code Division Multiple Access)}
%%%%%%%%%%%%%%%%%%%%%%%%%%%%%%%%%%%%%%%%%%%%%%%%%%%%%%%%%%%%%%
\subsection{OFDMA (Orthogonal Frequency Division Multiple Acess)}
%%%%%%%%%%%%%%%%%%%%%%%%%%%%%%%%%%%%%%%%%%%%%%%%%%%%%%%%%%%%%%
\subsection{WDM (Wavelength Division Multiplexing)}
%%%%%%%%%%%%%%%%%%%%%%%%%%%%%%%%%%%%%%%%%%%%%%%%%%%%%%%%%%%%%%
\subsection{DWDM (Dense Wavelength Division Multiplexing)}

\end{document}
