\documentclass[11pt]{beamer}
\usetheme{Madrid}
\usefonttheme{serif}

\usepackage[utf8]{inputenc}
% \usepackage[brazil]{babel}
\usepackage[T1]{fontenc}

\usepackage{amsmath}
\usepackage{amsfonts}
\usepackage{amssymb}
\usepackage{graphicx}

\usepackage{caption}
\usepackage{siunitx}

\DeclareMathOperator{\sen}{sen}
\DeclareMathOperator{\tg}{tg}

\setbeamertemplate{caption}[numbered]

\author[]{Diogo Fonseca \\ Professor: João Dias}
\title{From Waveforms to Bits}
% Informe o seu email de contato no comando a seguir
% Por exemplo, alcebiades.col@ufes.br
\newcommand{\email}{email}
%\setbeamercovered{transparent} 
\setbeamertemplate{navigation symbols}{} 
\logo{\includegraphics[scale=0.15]{imagens/ualg_tiny.png}}
\institute[]{UALG \par MESTRADO EM ENGENHARIA INFORMÁTICA} 
\date{8 de Novembro de 2025}
%\subject{}

% ---------------------------------------------------------
% Selecione um estilo de referência
\bibliographystyle{apalike}

%\bibliographystyle{abbrv}
%\setbeamertemplate{bibliography item}{\insertbiblabel}
% ---------------------------------------------------------

% ---------------------------------------------------------
% Incluir os slides nos quais as referências foram citadas
%\usepackage[brazilian,hyperpageref]{backref}

%\renewcommand{\backrefpagesname}{Citado na(s) página(s):~}
%\renewcommand{\backref}{}
%\renewcommand*{\backrefalt}[4]{
%	\ifcase #1 %
%		Nenhuma citação no texto.%
%	\or
%		Citado na página #2.%
%	\else
%		Citado #1 vezes nas páginas #2.%
%	\fi}%
% ---------------------------------------------------------

\begin{document}

\begin{frame}
	\titlepage
\end{frame}

\begin{frame}{Table of contents}
    \only<1>{\tableofcontents[sections={1-3}]}
    \only<2>{\tableofcontents[sections={4-6}]}
\end{frame}


%\begin{frame}[allowframebreaks]{Sumário}
%\tableofcontents 
%\end{frame}



\section{Theoretical Basis}

%%%%%%%%%%%%%%%%%%%%%%%%%%%%%%%%%%%%%%%%%%%%%%%%%%%%%%%%%%%%%%
\subsection{Fourier Analysis}

\begin{frame}
    \begin{beamercolorbox}[sep=8pt,center,rounded=true,shadow=true]{title}
        \usebeamerfont{title}\Huge\textbf{Fourier Analysis}
    \end{beamercolorbox}
\end{frame}

\begin{frame}{Fourier Analysis}
    \begin{block}{Fourier Series}
        $$ g(t) = \frac{1}{2}\boldsymbol{c} + \sum^{\infty}_{n=1} \boldsymbol{a_n} sin(2 \pi n f t) + \sum^{\infty}_{n=1} \boldsymbol{b_n} cos(2 \pi n f t)$$
    \end{block}

    \bigskip
    
	\only<2->{$ \boldsymbol{a_n} = \frac{2}{T} \int^{T}_{0} g(t) sin(2 \pi n f t) dt $}

    \bigskip
    
	\only<3->{$ \boldsymbol{b_n} = \frac{2}{T} \int^{T}_{0} g(t) cos(2 \pi n f t) dt $}

    \bigskip
    
	\only<4->{$ \boldsymbol{c} = \frac{2}{T} \int^{T}_{0} g(t)dt $}
\end{frame}







%%%%%%%%%%%%%%%%%%%%%%%%%%%%%%%%%%%%%%%%%%%%%%%%%%%%%%%%%%%%%%
\subsection{Bandwidth-Limited Signals}

\begin{frame}
    \begin{beamercolorbox}[sep=8pt,center,rounded=true,shadow=true]{title}
        \usebeamerfont{title}\Huge\textbf{Bandwidth-Limited Signals}
    \end{beamercolorbox}
\end{frame}

\begin{frame}{Bandwidth-Limited Signals}
	\begin{center}
		\color{structure}
		\bfseries\LARGE
		Some definitions
	\end{center}

	\vspace{1.0cm}

	\begin{itemize}
		\item Bandwidth (analog)
		\item Bandwidth (digital)
		\item Cutoff
		\item Baseband
		\item Passband
	\end{itemize}
\end{frame}

\begin{frame}{Bandwidth-Limited Signals}
    \begin{figure}[htb]
        \centering

        \includegraphics[width=0.40\textwidth]{imagens/harmonics.png}

		\captionsetup{font=scriptsize}
		\caption{(a) A binary signal and its root-mean-square Fourier amplitudes. (b)-(e) Successive approximations to the original signal.}
    \end{figure}
\end{frame}

\begin{frame}{Bandwidth-Limited Signals}
    \begin{table}[htb]
        \centering

		\begin{tabular}{|r|S|S|S|}
	        \hline
			\textbf{Bps} & \textbf{T (msec)} & \textbf{First harmonic (Hz)} & \textbf{\# Harmonics sent} \\ \hline
	        300		& 26.67	& 37.5	& 80	\\ \hline
	        600		& 13.33	& 75	& 40	\\ \hline
	        1200	& 6.67	& 150	& 20	\\ \hline
	        2400	& 3.33	& 300	& 10	\\ \hline
	        4800	& 1.67	& 600	& 5		\\ \hline
	        9600	& 0.83	& 1200	& 2		\\ \hline
	        19200	& 0.42	& 2400	& 1		\\ \hline
	        38400	& 0.21	& 4800	& 0		\\ \hline
        \end{tabular}

		\captionsetup{font=scriptsize}
		\caption{Relation between data rate and harmonics.}
    \end{table}
\end{frame}




%%%%%%%%%%%%%%%%%%%%%%%%%%%%%%%%%%%%%%%%%%%%%%%%%%%%%%%%%%%%%%
\section{The Maximum Data Rate of a Channel}

\begin{frame}
    \begin{beamercolorbox}[sep=8pt,center,rounded=true,shadow=true]{title}
        \usebeamerfont{title}\Huge\textbf{Maximum Data Rate of a Channel}
    \end{beamercolorbox}
\end{frame}

\begin{frame}{Maximum Data Rate of a Channel}
	Assume a perfect channel for communication with \textbf{no noise or interferance}.
	Furthermore, assume an arbitrary signal that has been run through a low-pass filter of bandwidth $\boldsymbol{B}$.
	\begin{block}<2->{Nyquist's sample rate proof}
		The filtered signal can be completely reconstructed by making exactly $\boldsymbol{2B}$ samples per second.
    \end{block}
	\only<3->{And given $\boldsymbol{V}$, the number of discrete states a signal can transmit at each sampling instant:}
	\begin{block}<4->{Nyquist's theorem}
		Maximum data rate of channel = $ \boldsymbol{2B \log_2(V)}$ bits/sec
    \end{block}
\end{frame}

\begin{frame}{Maximum Data Rate of a Channel}
	\centering
	\usebeamerfont{title}\Huge\textbf{What about interferance?}
\end{frame}

\begin{frame}{Maximum Data Rate of a Channel}
	Assume a channel analogous to the last, with a signal power $ \boldsymbol{S} $ and a noise power $ \boldsymbol{N} $.
	\begin{block}<2->{SNR (Signal-to-Noise Ratio)}
		$$ \frac{\boldsymbol{S}}{\boldsymbol{N}} $$
    \end{block}
	\begin{block}<3->{Capacity of a real channel}
		Maximum data rate of channel = $ \boldsymbol{B \log_2(1 + \frac{S}{N})}$ bits/sec
    \end{block}
\end{frame}






%%%%%%%%%%%%%%%%%%%%%%%%%%%%%%%%%%%%%%%%%%%%%%%%%%%%%%%%%%%%%%
\section{Digital Modulation}

\begin{frame}
    \begin{beamercolorbox}[sep=8pt,center,rounded=true,shadow=true]{title}
        \usebeamerfont{title}\Huge\textbf{Digital Modulation}
    \end{beamercolorbox}
\end{frame}



%%%%%%%%%%%%%%%%%%%%%%%%%%%%%%%%%%%%%%%%%%%%%%%%%%%%%%%%%%%%%%
\subsection{Baseband Transmission}

\begin{frame}
    \begin{beamercolorbox}[sep=8pt,center,rounded=true,shadow=true]{title}
        \usebeamerfont{title}\Huge\textbf{Baseband transmission}
    \end{beamercolorbox}
\end{frame}

\begin{frame}{Baseband Transmission}
	\textbf{NRZ scheme} (Non-Return-to-Zero)
	\begin{itemize}
		\item<2-> Signal follows the data. (i.e. Positive voltage for one, negative voltage for zero)
		\item<2-> Sender sends the signal.
		\item<2-> Receiver samples the signal at regular intervals of time.
		\item<2-> Receiver maps signal samples to the closes symbols (i.e. $\{0, 1\}$)
	\end{itemize}
	\includegraphics[width=1.00\textwidth]{imagens/NRZ.png}<3->
\end{frame}

%%%%%%%%%%%%%%%%%%%%%%%%%%%%%%%%%%%%%%%%%%%%%%%%%%%%%%%%%%%%%%
\subsection{Bandwidth Efficiency}

\begin{frame}
    \begin{beamercolorbox}[sep=8pt,center,rounded=true,shadow=true]{title}
        \usebeamerfont{title}\Huge\textbf{Bandwidth Efficiency}
    \end{beamercolorbox}
\end{frame}

\begin{frame}{Bandwidth Efficiency}
	\textbf{NRZ scheme}
	\begin{itemize}
		\item<2-> For a data rate of $ B $ bits/sec we need a bandwidth of at least $ \boldsymbol{\frac{B}{2}} $ Hz.
		\item<3-> Given this is a fundamental limit, we cannot run NRZ faster without using additional bandwidth.
		\item<4-> One strategy to increase efficiency is to use more than two signaling levels.
		\item<5-> By using 4 different voltages, for instance, we can send 2 bits at once as a single \textbf{symbol}.
		\item<6-> This works as long as the signal at the receiver is sufficiently strong to distinguish all the levels.
		\item<7-> The rate at which the signal changes (\textbf{symbol rate}) is then half the bit rate, so bandwidth has been reduced.
		\item<8-> Bitrate can be interpreted as \textbf{(symbol rate $ \boldsymbol{\times} $ bits per symbol)}.
	\end{itemize}
\end{frame}


%%%%%%%%%%%%%%%%%%%%%%%%%%%%%%%%%%%%%%%%%%%%%%%%%%%%%%%%%%%%%%
\subsection{Clock Recovery}

\begin{frame}
    \begin{beamercolorbox}[sep=8pt,center,rounded=true,shadow=true]{title}
        \usebeamerfont{title}\Huge\textbf{Clock Recovery}
    \end{beamercolorbox}
\end{frame}

\begin{frame}{Clock Recovery}
	\begin{center}
		\color{structure}
		\bfseries\LARGE
		The problem
	\end{center}

	\begin{itemize}
		\item<2-> The receiver must know when one symbol ends and the next symbol appears.
		\item<3-> After a while it's hard to tell the bits apart, 15 zeroes look much like 16 zeroes unless you have a very accurate clock.
		\item<4-> Very accurate clocks are expensive.
	\end{itemize}
\end{frame}

\begin{frame}{Clock Recovery}
	\centering
	\textbf{Manchester encoding}

	\bigskip

	\includegraphics[width=1.00\textwidth]{imagens/bit_stream.png}
	\includegraphics[width=1.00\textwidth]{imagens/manchester_encoding.png}

	\begin{itemize}
		\item<2-> Requires twice the bandwidth.
		\item<2-> Used in classic Ethernet.
	\end{itemize}
\end{frame}

\begin{frame}{Clock Recovery}
	\centering
	\textbf{NRZI (Non-Return-to-Zero Inverse)}

	\bigskip

	\includegraphics[width=1.00\textwidth]{imagens/bit_stream.png}
	\includegraphics[width=1.00\textwidth]{imagens/NRZI.png}

	\begin{itemize}
		\item<2-> Long streaks of 0 still have the same problem.
		\item<2-> Used in USB.
	\end{itemize}
\end{frame}

\begin{frame}{Clock Recovery}
	\centering
	\textbf{4B/5B}

	\bigskip

    \begin{table}[htb]
        \centering

		\begin{tabular}{|l|l|l|l|}
	        \hline
			\textbf{Data (4B)} & \textbf{Codeword (5B)} & \textbf{Data (4B)} & \textbf{Codeword (5B)} \\ \hline
	        0000	& 11110	& 1000	& 10010	\\ \hline
	        0001	& 01001	& 1001	& 10011	\\ \hline
	        0010	& 10100	& 1010	& 10110	\\ \hline
	        0011	& 10101	& 1011	& 10111	\\ \hline
	        0100	& 01010	& 1100	& 11010	\\ \hline
	        0101	& 01011	& 1101	& 11011	\\ \hline
	        0110	& 01110	& 1110	& 11100	\\ \hline
	        0111	& 01111	& 1111	& 11101	\\ \hline
        \end{tabular}

		\captionsetup{font=scriptsize}
		\caption{4B/5B mapping.}
    \end{table}

	\begin{itemize}
		\item<2-> 25\% overhead.
	\end{itemize}
\end{frame}

\begin{frame}{Clock Recovery}
	\begin{center}
		\textbf{Scrambling}
	\end{center}

	\bigskip

	Scrambles the data by XORing it with a pseudorandom sequence. Receiver XORs the data with the same sequence.

	\bigskip

	\begin{itemize}
		\item<2-> Used in early versions of IP over SONET.
		\item<2-> Still possible to get long streaks of identical symbols.
		\item<2-> Possible malicious usage, "killer packets".
	\end{itemize}
\end{frame}

%%%%%%%%%%%%%%%%%%%%%%%%%%%%%%%%%%%%%%%%%%%%%%%%%%%%%%%%%%%%%%
\subsection{Balanced Signals}

\begin{frame}
    \begin{beamercolorbox}[sep=8pt,center,rounded=true,shadow=true]{title}
        \usebeamerfont{title}\Huge\textbf{Balanced Signals}
    \end{beamercolorbox}
\end{frame}

\begin{frame}{Balanced Signals}
	\centering
	\textbf{Bipolar Encoding}

	\bigskip

	\includegraphics[width=1.00\textwidth]{imagens/bit_stream.png}
	\includegraphics[width=1.00\textwidth]{imagens/bipolar_encoding.png}

	\begin{itemize}
		\item<2-> Adds a voltage level.
		\item<2-> Average of 0.
		\item<2-> Indirectly helps with clock recovery.
	\end{itemize}
\end{frame}

\begin{frame}{Balanced Signals}
	\centering
	\textbf{4B/5B (again) or 8B/10B}

	\bigskip

	\begin{itemize}
		\item<2-> 8B/10B has at most a disparity of 2 bits.
		\item<2-> 8B/10B is 80\% efficient.
		\item<2-> Helps with clock recovery.
	\end{itemize}
\end{frame}

%%%%%%%%%%%%%%%%%%%%%%%%%%%%%%%%%%%%%%%%%%%%%%%%%%%%%%%%%%%%%%
\subsection{Passband Transmission}

\begin{frame}
    \begin{beamercolorbox}[sep=8pt,center,rounded=true,shadow=true]{title}
        \usebeamerfont{title}\Huge\textbf{Passband Transmission}
    \end{beamercolorbox}
\end{frame}

\begin{frame}{Passband Transmission}
	\centering
	\textbf{ASK (Amplitude Shift Keying)}

	\bigskip

	\includegraphics[width=1.00\textwidth]{imagens/bit_stream_passband.png}
	\includegraphics[width=1.00\textwidth]{imagens/ask.png}
\end{frame}

\begin{frame}{Passband Transmission}
	\centering
	\textbf{FSK (Frequency Shift Keying)}

	\bigskip

	\includegraphics[width=1.00\textwidth]{imagens/bit_stream_passband.png}
	\includegraphics[width=1.00\textwidth]{imagens/fsk.png}
\end{frame}

\begin{frame}{Passband Transmission}
	\centering
	\textbf{PSK (Phase Shift Keying)}

	\bigskip

	\includegraphics[width=1.00\textwidth]{imagens/bit_stream_passband.png}
	\includegraphics[width=1.00\textwidth]{imagens/psk.png}
\end{frame}

\begin{frame}{Passband Transmission}
	\centering
	\textbf{QPSK (Quadrature Phase Shift Keying)}

	\bigskip

	\includegraphics[width=0.5\textwidth]{imagens/qpsk.png}
\end{frame}

\begin{frame}{Passband Transmission}
	\centering
	\textbf{QAM-16 (Quadrature Amplitude Modulation - 16)}

	\bigskip
	\includegraphics[width=0.5\textwidth]{imagens/qam16.png}
\end{frame}

\begin{frame}{Passband Transmission}
	\centering
	\textbf{QAM-64 (Quadrature Amplitude Modulation - 64)}

	\bigskip

	\includegraphics[width=0.5\textwidth]{imagens/qam64.png}
\end{frame}

\begin{frame}{Passband Transmission}
	\centering
	\textbf{Gray-coded QAM-16}

	\bigskip

	\includegraphics[width=0.9\textwidth]{imagens/gray_coded_qam16.png}
\end{frame}

%%%%%%%%%%%%%%%%%%%%%%%%%%%%%%%%%%%%%%%%%%%%%%%%%%%%%%%%%%%%%%
\section{Multiplexing}

\begin{frame}
    \begin{beamercolorbox}[sep=8pt,center,rounded=true,shadow=true]{title}
        \usebeamerfont{title}\Huge\textbf{Multiplexing}
    \end{beamercolorbox}
\end{frame}


%%%%%%%%%%%%%%%%%%%%%%%%%%%%%%%%%%%%%%%%%%%%%%%%%%%%%%%%%%%%%%
\subsection{FDM (Frequency Division Multiplexing)}

\begin{frame}{FDM (Frequency Division Multiplexing)}
	\centering

	\begin{itemize}
		\item Divides the spectrum into frequency bands.
		\item Requires guard bands.
		\item Used by AM radio, telephone networks, cellular, terrestrial wireless and satelite networks.
	\end{itemize}

	\includegraphics[width=0.75\textwidth]{imagens/fdm.png}<2->
\end{frame}

%%%%%%%%%%%%%%%%%%%%%%%%%%%%%%%%%%%%%%%%%%%%%%%%%%%%%%%%%%%%%%
\subsection{OFDM (Orthogonal Frequency Division Multiplexing)}

\begin{frame}{OFDM (Orthogonal Frequency Division Multiplexing)}
	\centering

	\begin{itemize}
		\item Channel bandwidth is divided into many independent subcarriers (i.e. QAM).
		\item Subcarriers are packed tightly in the frequency domain.
		\item Frequency response of each subcarrier is designed to be zero at the center of adjacent subcarriers.
		\item Guard time is needed to repeat a portion of the signals.
		\item Used in 802.11, cable networks, power-line networking, and 4G cellular systems.
	\end{itemize}

	\includegraphics[width=0.75\textwidth]{imagens/ofdm.png}
\end{frame}

%%%%%%%%%%%%%%%%%%%%%%%%%%%%%%%%%%%%%%%%%%%%%%%%%%%%%%%%%%%%%%
\subsection{TDM (Time Division Multiplexing)}

\begin{frame}{TDM (Time Division Multiplexing)}
	\centering

	\begin{itemize}
		\item Users take turns to share the whole bandwidth (round-robin).
		\item Bits from each input stream are taken in a fixed time slot.
		\item Input stream is output to an aggregate stream.
		\item Used in telephone and celullar networks.
	\end{itemize}

	\includegraphics[width=0.75\textwidth]{imagens/tdm.png}
\end{frame}

%%%%%%%%%%%%%%%%%%%%%%%%%%%%%%%%%%%%%%%%%%%%%%%%%%%%%%%%%%%%%%
\subsection{STDM (Statistical Time Division Multiplexing)}

\begin{frame}{STDM (Statistical Time Division Multiplexing)}
	\begin{itemize}
		\item Almost analogous to TDM.
		\item No fixed schedule.
		\item Schedule is decided based on usage information.
	\end{itemize}
\end{frame}

%%%%%%%%%%%%%%%%%%%%%%%%%%%%%%%%%%%%%%%%%%%%%%%%%%%%%%%%%%%%%%
\subsection{CDM/CDMA (Code Division Multiplexing / Code Division Multiple Access)}

\begin{frame}{CDM/CDMA (Code Division Multiplexing / Code Division Multiple Access)}
	\centering

	\begin{itemize}
		\item Very wide frequency band
		\item Users share the whole frequency spectrum at all time.
		\item $ \boldsymbol{m} $ unique chip vectors $ \boldsymbol{S} $ are chosen for each station (generated with Walsh codes).
		\item each chip vector $ \boldsymbol{S} $ represents a 1, and it's complement $ \boldsymbol{\bar{S}} $ represents a 0.
	\end{itemize}

	\includegraphics[width=0.75\textwidth]{imagens/cdma.png}
\end{frame}

%%%%%%%%%%%%%%%%%%%%%%%%%%%%%%%%%%%%%%%%%%%%%%%%%%%%%%%%%%%%%%
\subsection{OFDMA (Orthogonal Frequency Division Multiple Access)}

\begin{frame}{OFDMA (Orthogonal Frequency Division Multiple Access)}
	\begin{itemize}
		\item
	\end{itemize}
\end{frame}

%%%%%%%%%%%%%%%%%%%%%%%%%%%%%%%%%%%%%%%%%%%%%%%%%%%%%%%%%%%%%%
\subsection{WDM (Wavelength Division Multiplexing)}

\begin{frame}{WDM (Wavelength Division Multiplexing)}
	\centering

	\begin{itemize}
		\item Basically just FDM but for very high frequencies (optical fiber)
	\end{itemize}

	\includegraphics[width=0.75\textwidth]{imagens/wdm.png}
\end{frame}

%%%%%%%%%%%%%%%%%%%%%%%%%%%%%%%%%%%%%%%%%%%%%%%%%%%%%%%%%%%%%%
\subsection{DWDM (Dense Wavelength Division Multiplexing)}

\begin{frame}{DWDM (Dense Wavelength Division Multiplexing)}
	Analogous to DWDM but higher number of channels and little space between each channel.
\end{frame}


\end{document}
