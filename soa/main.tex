%\documentclass{cwpreport} % for review/editing/polishing
%\linespread{1.5}           % for review/editing/polishing
\documentclass[onecolumn]{cwpreport} % for final draft


% Packages used by default CWP reports built by M8R
\usepackage{times,natbib,amsmath,graphicx,color,amssymb,amsbsy,lineno,setspace,algorithm2e,lipsum,booktabs}
\bibliographystyle{plainnat}

% Specific paper formatting for CWP report
\setlength{\paperwidth}{8.5in}
\setlength{\paperheight}{11.0in}
\setlength{\topmargin}{-0.25in}
\setlength{\textheight}{8.75in}
\setlength{\textwidth}{6.5in}
\setlength{\oddsidemargin}{+.015625in}
\setlength{\evensidemargin}{+.015625in}

% Final draft only
\title[]{State of Art on CDMA interference mitigation technology} 
% for editing/reviewing
%\title{Measuring wave propagation through an open-pit mine using stereo videos} 
%\righthead{training data}
%\lefthead{Rapstine & Sava}

% Final draft only
\author[]{Diogo Fonseca
\\
Masters in Engenharia Informática, University of Algarve} 
%
%\author{Thomas Rapstine, Paul Sava, Ashley Grant, \& Jeff Shragge}
\begin{document}

\maketitle

% Uncomment for final draft only
%\setcounter{page}{0}
\journal{}

%% ------------------------------------------------------------
\begin{abstract}
\lipsum[2]
\end{abstract}
%% ------------------------------------------------------------

% Uncomment for final draft only
%% ------------------------------------------------------------
% \begin{keywords}
% Keyword1, Keyword2, Keyword3 
% \end{keywords}
%% ------------------------------------------------------------

%% ------------------------------------------------------------
\section{Introduction}

Code-division multiple access (CDMA) is an algorithm for multiple users to use the same channel simultaneously. The way
it works is by assigning different orthogonal codes (chips) for each user in such a way that it allows them to represent a 1 or
a 0. A 1 is then represented by the original assigned orthogonal code for that user, and 0 represented by its simetric.
CDMA was often used for 2G and 3G mobile networks, and is still used for GPS and military communication.

Because the codes are orthogonal, when they merge it's possible to recover each and every user's message from the joint waves.
This is only possible, however, if the waves merge at the exact same phase on the receiver end. Furthermore if one wave from
one transmitter/user has higher amplitude (is "louder") than the others, it drowns the others and makes them unrecoverable.
This state-of-art report focuses on how modern CDMA and variants deal with these and some other multiple-access
interference (MAI) problems.

% Here is an example of an citation \citep{hale_atomic_2001}. Here is an example of an inline citation from \cite{haber_inversion_2007}.

%% ------------------------------------------------------------

\section{MAI (multiple-access interference)}

\subsection{Near-Far problem}
The near-far problem is the single most fundamental flaw of CDMA. Starting with an example,
let's say we have 4 bit chips for each user as such:
\begin{itemize}
	\item \textbf{User 1:} $\boldsymbol{[+1, -1, +1, +1]}$
	\item \textbf{User 2:} $\boldsymbol{[+1, +1, -1, -1]}$
	\item \textbf{User 3:} $\boldsymbol{[-1, +1, +1, -1]}$
\end{itemize}

ideally, when the waves constructively and destructively interfere, the first bit of the chip will result in $ (1 + 1 - 1) = 2 $.
However, due to different distancing, or even different powered RF transmitters, this results in different
magnitudes of waves being sent. What should be $ 1 + 1 - 1 $ might (anecdotally) actually result in $ 2 + 0.1 - 5 = -2.9$ due to
the distance and power at which the transmitters are broadcasting.

This destroys the orthogonality of the space, making it impossible to decode the original bits from chips for \textbf{every}
user. This shows that for vector orthogonality to be achieved, it is mandatory to have uniform amplitudes for the waves.
The price for the noncompliance of uniformity being the loss of orthogonality, and thus the loss of information.

\subsection{Out-of-phase chips}
This is the second most substantial problem in CDMA. Also potentially resulting in a complete loss of information.
The problem relies in the way the waves constructively and destructively interfere. If the waves are interfering out
of phase, this will result in loss of orthogonality. An example follows, analogous to the above.

\begin{itemize}
	\item \textbf{User 1:} $\boldsymbol{[+1, -1, +1, +1]}$
	\item \textbf{User 2:} $\boldsymbol{[+1, +1, -1, -1]}$
	\item \textbf{User 3:} $\boldsymbol{[-1, +1, +1, -1]}$
\end{itemize}

Imagine that \textit{User 3} either moved in space in relation to the receiving station, has a slight processing delay, or any
other problem resulting in a 1-chip delay.
Instead of the waves interfering as such: $(1 + 1 - 1) = 2$, they will interfere in the following way: $(1 + 1 + 1) = 3$.
The same applies for all the following chips, resulting in loss of orthogonality, and therefore total loss of
data. The same fenomena occurrs for half-bit delays and potentially for even smaller delays. The same obviously
also applies for greater chip phase shifts (delays).


\subsection{Multipath channel distortion}

Multipath distortion occurrs when phase-shifted, lower amplitude replicas of the same wave
reach the receiver. This is often the product of waves bouncing on surfaces and reaching
the receiver as "echoes", often at later times (phase shifted) and with increasingly lower amplitudes.

As this is not random noise or interference and as such classifies as structured interference, it strongly impacts
the decoding process. We are getting "phantom" versions of valid chips, interfering with the rest in such a way
that it might cripple the decoding process. The more multipath channel distortion, the worse the decoding error becomes.

\subsection{Doppler effect}

As transmitters move in space, their transmitted waves are subject to the doppler effect, making them
higher frequency in the direction of movement and lower frequency in the opposite direction.
This creates a continuously changing phase shift, which creates a special case of the out-of-phase
chips in which the phase keeps changing, making it seem as if the signal is "spinning" on the
receiver end.

%% ------------------------------------------------------------

\section{Problem statements}

\subsection{Downlink (synchronous CDMA)}

The problem downlink, meaning that we want to send data from the central tower to the users
is pretty straight forward, we want to broadcast information in the channel so that all users
can decode their part of the information. Amplitude of the signal is dictated by the central
station and therefore should be "audible" by all users in range. Furthermore there is no
MAI to worry about, as long as the user is close enough to a central station in such a way
that it's amplitude overpowers other potential stations using CSMA. In other words, this
usually means that the distance from the user to one of the stations is lower than others,
making MAI not a problem for user decoding.

\subsection{Uplink (asynchronous CDMA)}

The problem uplink, meaning that we want to send data from the users to the central tower
is much more complicated than the downlink. It implies that all users, with different
locations, different amplitude transmitters, different speeds and different distances
from the central tower, need to communicate with the central tower.

This, of course, leads to an amount of MAI that makes the data completely incoherent
and impossible to decode by the central station. This is the core limitation of
CDMA, and where all the aspects of MAI really come into play in a very destructive way.

%% ------------------------------------------------------------

\section{Solutions}

\subsection{CDMA Rake}

The CDMA Rake solution was primarily designed to deal with the multipath channel distortion, but
ends up mitigating a lot of the MAI problems.
The way it manages this, is by first replacing the orthognal codes, with quasi-orthogonal codes
called PN codes.

\subsubsection{PN codes}

Assume each user has a unique identification (i.e. A phone's electronic serial number (ESN)).
Furthermore, assume the base station has already communicated with the user, and gotten it's
unique identification. Upon this, usually based on both the current time and unique identification
of the user, using a standard predefined algorithm, an identical seed is generated for the user and
base station. Upon the base station and user having the same seed, both then generate a (usually) $ 2^{42} $-long
pseudorandom noise (PN) code based on it. Since both have the same seed, and therefore the same PN code,
now they're ready to begin CDMA communication.

This PN code is uniformly random, in such a way that the average of all values are 0 (if each bit
is encoded between either -1 or 1). This both makes time synchronization easy and makes it that so
other parallel PN codes that are being transmitted are equivalent to noise, no longer being structured
interferance. This way it's as if the chips vectors are orthognal. The way bits are then encoded in this
PN code is as follows: if the chipp sequence length is $ n $, that means each data bit will be encoded into
the chipp sequence

$$ C_i = (C_{i \cdot n}, C_{i \cdot n + 1}, \ldots,  C_{(i + 1) \cdot n - 1}) $$
and a 0 data bit will be encoded to $ -C_i $. The next data bit sent, will already be encoded into
$ C_{i + 1} $, and so on and so forth.

\subsubsection{The Rake System}

The rake consists of multiple parallel processing units (called "Fingers"), a controller, and a combiner.
Each finger receives the signal and correlates it with phase-shifted versions of the user's PN code (the possible "echoes").
The controller takes all of these signals and searches through all possible delays of the channel, called the
"delay spread" of the channel, in order to identify peaks in the correlation profile, finding each echo.
The combiner takes these echoes, phase-shifts them such that they all coincide, and they are all summed, acting as gain, amplifying the
amplitude of the received signal. Each echo is asigned a different weight proportional to it's amplitude and correlation profile
(how probable is it to be an actual echo of the same signal) such that the sum is actually a weighted sum of these signals/echoes.

\subsubsection{MAI mitigation}

The more fingers, the better the rake system becomes, but it also increases power consumption. In this way, the Rake receiver turns a
known weakness into a strength, exploiting the signal replicas and turning them into gain.
As such, it severely mitigates multipath channel distortion making it into a non-problem. 

Not only that, but the use of PN codes mitigates the out-of-phase chips, as the chips from other senders will cancel out into noise,
making each chip orthogonal with the background noise. This gets mitigated to such a degree that out-of-phase chips also stop being a problem.

\subsection{Space-time coded CDMA systems}

The main idea behind space-time coding (STC) is to use multiple antennas to transmit. This introduces structured redundancy in both space and time.
In such a way that we get signal gain by listening to all the antennas, or we can still decode messages when other antennas are obstructed.

The way this works is by having each antenna broadcast the same information but at different times with order information.
Following is an example of how this can be done for two antennas:

\begin{tabular}{|c|c|c|}
\toprule
& \textbf{Antenna 1} & \textbf{Antenna 2} \\
\midrule
\textbf{t=0} & & \\
\textbf{t=1} & & \\
\bottomrule
\end{tabular}

\subsection{Multiuser detectors}

\subsection{Common downlink solution}

\subsection{Common uplink solution}

%% ------------------------------------------------------------

\section{Conclusions}

% \begin{figure}
%     \centering
%     \includegraphics[width=0.475\textwidth]{Fig/DaintreeRiver.jpg}  
%     \includegraphics[width=0.475\textwidth]{Fig/GeledaMonkeys.jpg} 
%     \caption{Photo of the (a) Daintree River in Australia and (b) the Geleda Monkeys.}
%     \label{fig:model_volume}
% \end{figure}

%% ------------------------------------------------------------
\newpage
\nocite{*}
\bibliography{references}
%% ------------------------------------------------------------

%% ------------------------------------------------------------


%% -----------------
\end{document}
